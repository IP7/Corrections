%pour generer et afficher le pdf en ligne de commande : pdflatex td1.tex && evince td1.pdf
\documentclass[a4paper]{article}

\usepackage[utf8]{inputenc}
\usepackage[a4paper]{geometry}
%\usepackage[]{algorithm2e}
\usepackage{amsfonts}
\usepackage{amsmath}
\usepackage{mathtools}
\usepackage[francais]{babel}

\title{Correction du TD1 d'Infographie}
\author{Thomas Salmon}
\date{07/02/2014}

\begin{document}
\maketitle
\section{Geometrie du plan}
\subsection{Exercice 1}
\subsubsection{a)}
\begin{math}  
  A \equiv P_0$, $ \overrightarrow{U} \equiv \overrightarrow{P_0P_1} \\  
  $conditon$ \overrightarrow{P_0 P_1} \propto \overrightarrow{P_0M} = 0
\end{math}
\subsubsection{b)}
\begin{math}
  P_0 = (x_0, y_0)$, $ P_1 = (x_1, y_1)$, $ M = (x, y) \\
  \begin{pmatrix}
    x_1 - x_0\\
    y_1 - y_0
  \end{pmatrix}
  \propto
  \begin{pmatrix}
    x - x_0\\
    y - y_0
  \end{pmatrix}
  = 0 \\
  (x_1 - x_0)(y - y_0) - (y_1 - y_0)(x - x_0) = 0 \\
  ax + by + c = 0 \\
  \\
  a = -(y_1 - y_0), b = (x_1 - x_0)\\
  \rightarrow \overrightarrow{P_1P_0} =  
  \begin{pmatrix}
    b\\
    a
  \end{pmatrix}
\end{math}
\subsubsection{c)}
\[ 
\left\{
\begin{array}{r c l}
  \overrightarrow{P_0P_1} & \propto & \overrightarrow{P_0M} \\
  \overrightarrow{P_0P_1} & \propto & \overrightarrow{P_0M'}
\end{array}
\right .
\]
Les deux sont de meme signe.
\subsubsection{d}

\subsection{Exercice 2}  
vecteurs directeurs
\begin{math}
  \begin{pmatrix}
    b\\
    -a
  \end{pmatrix}
  $ (cf 1.b)$\\
  ax + by + c = 0 \\
  b = 0 \Rightarrow $ la droite est celle des ordonnées, et le x commun est calculable$ \\ 
  a = 0 \Rightarrow $ la droite est celle des abscisses, et le y commun est calculable$ \\
\end{math}
\begin{description}
\item si $a \neq 0$:
\[ 
\begin{pmatrix}
  \frac{-c}{a}\\
  0
\end{pmatrix}
\left\{
\begin{array}{l}
  $pour $ y = 0 $ , on a $ ax + c = 0 \\
  $et pour $ x = \frac{-c}{a}
\end{array}
\right .
\]
\item si $b \neq 0$:
\[ 
\begin{pmatrix}
  0\\
  \frac{-c}{b}
\end{pmatrix}
\left\{
\begin{array}{l}
  $pour $ x = 0 $ , on a $ by + c = 0 \\
  $et pour $ y = \frac{-c}{b}
\end{array}
\right .
\]
\end{description}
\subsection{Exercice 3}
On calcul (1.b) 
\begin{description}
\item l'équation de la droite $(AB)$
\item l'équation de la droite $(CD)$
\end{description}
On obtient un systeme linéaire que l'on résoud par Pivot de Gauss\\
On obtient un point d'intersection $I$ et l'on vérifie que $I$ appartienne à $[AB]$ et à $[CD]$ (1.d)
\section{Partie II}
\subsection{Exercice 2}
\subsubsection{a}
\begin{description}
\item 2 dim : $\overrightarrow{U} \propto \overrightarrow{AM} \Leftrightarrow M \in \mathcal{D}$
\item 3 dim : $\overrightarrow{N} \circ \overrightarrow{AM} \Leftrightarrow M \in \mathcal{P}$
\end{description}
\subsubsection{b}
\begin{math}
\begin{pmatrix}
  N_x\\
  N_y\\
  N_z
\end{pmatrix}
\circ
\begin{pmatrix}
  A_x - x\\
  A_y - y\\
  A_z - z
\end{pmatrix}
= N_x(A_x -x) + N_y(A_y + y) + N_z(A_z + z) = 0 $  (equation du plan)$
\end{math}
\subsubsection{c}
meme question que b: \\
$\overrightarrow{N} \circ \overrightarrow{AM}$ et $\overrightarrow{N} \circ \overrightarrow{AM}$ de meme signe
\subsubsection{d}
dim 2 : \\
\begin{math}
\overrightarrow{AB} \propto \overrightarrow{AI} = 0\\
\overrightarrow{AB} \propto (\overrightarrow{AM} + \overrightarrow{MI}) = 0\\
\overrightarrow{AB} \propto (\overrightarrow{AM} + k\overrightarrow{U}) = 0
\end{math}

dim 3 : \\
\begin{math}
  \overrightarrow{N} \circ \overrightarrow{AI} = 0\\
  \overrightarrow{AB} \circ (\overrightarrow{AM} + \overrightarrow{MI}) = 0\\
  \overrightarrow{AB} \circ (\overrightarrow{AM} + k\overrightarrow{U}) = 0
\end{math}

\begin{math}
  \overrightarrow{N} \circ \overrightarrow{AM} + \overrightarrow{N} \circ k\overrightarrow{U}\\
  \overrightarrow{N} \circ \overrightarrow{AM} + k(\overrightarrow{N} \circ \overrightarrow{U})\\
\end{math}
\end{document}
